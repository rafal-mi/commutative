\documentclass{article}
\usepackage[utf8]{inputenc}
\usepackage{natbib}
\usepackage{graphicx}
\usepackage{tikz-cd}
\usepackage{amsmath}
\usepackage{amssymb}
\usepackage{yfonts}
\usepackage{hyperref}
\usepackage{amsthm}
\usepackage{stmaryrd}
\usepackage{mathtools}

\usepackage{geometry}
\geometry{
 a4paper,
 total={170mm,257mm},
 left=35mm,
 right=35mm,
 top=20mm,
 }

\hypersetup{
    colorlinks=true,
    linkcolor=blue,
    filecolor=magenta,      
    urlcolor=cyan,
}

\urlstyle{same}
%\newcommand{\goth}[1]{\text{\textgoth{#1}}

\newtheorem{theorem}{Isomorphism}

\title{Rings of Fractions Wallpaper}

\begin{document}

\maketitle

\begin{theorem}
Exercise 2.2
\end{theorem}

\[
A/\textgoth{a} \otimes_A M \simeq M/\textgoth{a}M
\]
\[
(a + \textgoth{a}) \otimes x \mapsto ax + \textgoth{a}M
\]

\vspace{10px}
\begin{theorem}
Corollary 3.4.iii. As $S^{-1}A$-modules
\end{theorem}

\[
S^{-1}(M/N) \simeq S^{-1}M/S^{-1}M
\]
\[
(m + N) / s \mapsto m/s + S^{-1}N 
\]
\noindent
but only after identification $S^{-1}\iota(S^{-1}N)) == S^{-1}N$ where $\iota$ is the inclusion of $N$ into $M$.

\vspace{10px}
\begin{theorem}
Proposition 3.5. As $S^{-1}A$-modules
\end{theorem}

\[
S^{-1}A \otimes_A M \simeq S^{-1}M
\]
\[
a/s \otimes m \mapsto am/s
\]
\noindent
and this is the general element (no sum).

\vspace{10px}
\begin{theorem}
Proposition 3.5 for $A_\textgoth{p}$. As $A_\textgoth{p}$-modules

\end{theorem}

\[
A_\textgoth{p} \otimes_A M \simeq M_\textgoth{p}
\]
\[
a/s \otimes m \mapsto am/s
\]
\noindent
and this is the general element (no sum).

\vspace{10px}
\begin{theorem}
Proposition 3.7. As $S^{-1}A$-modules
\end{theorem}

\[
S^{-1}M \otimes_{S^{-1}A} S^{-1}N \simeq  S^{-1}(M \otimes_A N)
\]
\[
m/s \otimes n/t \mapsto (m \otimes n)/st
\]

\vspace{10px}
\begin{theorem}
Proposition 3.7 for $A_\textgoth{p}$. As $A_\textgoth{p}$-modules
\end{theorem}

\[
M_\textgoth{p} \otimes_{A_\textgoth{p}} N_\textgoth{p} \simeq  (M \otimes_A N)_\textgoth{p}
\]
\[
m/s \otimes n/t \mapsto (m \otimes n)/st
\]



\end{document}