\documentclass{article}
\usepackage[utf8]{inputenc}
\usepackage{natbib}
\usepackage{graphicx}
\usepackage{tikz-cd}
\usepackage{tikz}
\usetikzlibrary{cd}
\usepackage{amsmath}
\usepackage{amssymb}
\usepackage{yfonts}
\usepackage{hyperref}
\usepackage{mathtools}
\usepackage{amsthm}
\usepackage{geometry}
\usepackage{stmaryrd}
\usepackage{subcaption}

\geometry{
 a4paper,
 total={170mm,257mm},
 left=35mm,
 right=35mm,
 top=20mm,
 }

\hypersetup{
    colorlinks=true,
    linkcolor=blue,
    filecolor=magenta,      
    urlcolor=cyan,
}

\urlstyle{same}
%\newcommand{\goth}[1]{\text{\textgoth{#1}}
\newcommand{\mf}{\mathfrak}
\newcommand{\aaa}{\mf a}
\newcommand{\bbb}{\mf b}
\newcommand{\ppp}{\mf p}
\newcommand{\qqq}{\mf q}
\newcommand{\Spec}{\operatorname{Spec}}

\title{Facts about Rings of Fractions}
\author{Rafal Michalski}
\date{}

\newtheorem{theorem}{Fact}[section]
\newtheorem{example}[theorem]{Example}

\begin{document} 

\maketitle

\section{Introduction}

\begin{theorem}
If $0 \in S$, then $S^{-1}A$ is a trivial ring.
\end{theorem}

\noindent
\textit{Proof.} Any $(a, s), (a', s')$ are related because $(as' - a's) \cdot 0 = 0$ with $0 \in S$. \

\bigskip

\begin{theorem}
$A$ a PID, the equivalence relation in $ A \times S $ is: $ (a, s) \equiv (a', s') $ iff $ as' = a's $. \qed
\end{theorem}

\bigskip
\begin{theorem}
For $A$ a field, and $S = \{-1, 1\}$, $S^{-1}A \cong A$.
\end{theorem}

\noindent
\textit{Proof.} It is easily verified that the standard isomorphism from $A$ to $S^{-1}A$ is 1-1 and onto. \qed

\bigskip
\begin{theorem}
For $A$ a field, and $S$ a multiplicatively closed subset of $A$ not containing zero, $S^{-1}A \cong A$.
\end{theorem}

\noindent
\textit{Proof.} The standard homomorphism $f:a \mapsto a/1$ of $A$ into $S^{-1}A$ is injectve: if $a/1 = a'/1$ then $a \cdot 1 = a1 \cdot 1$, then $a = a'$. It is surjective: $f(as^{-1}) = f(a)f(s^{-1}) = (a/1)(s^{-1}/1) = \ldots$, but $s^{-1}/1 = 1/s$ as $s^{-1}s = 1 \cdot 1$; continuing, $\ldots = (a/1)(1/s) = a/s$. \qed

\bigskip
\begin{theorem}
For $A$ a field, the ring of fractions and the field of fractions are isomorphic.
\end{theorem}

\noindent
\textit{Proof.} For isomorphism of $A$ with its field of fractions, see Math Exchange 79188. About the isomorphism with its ring of fractions, is the fact above. \qed

\bigskip
\begin{example}
Some example.
\end{example}

\vspace{.2em}
\begin{theorem}
The quotient ring $A/I$ can be viewed as an $A$-module, and then the ring of fractions $T^{-1}(A/I)$, where $T$ is the image of $S$ in $A/I$, equals the module of fractions $S^{-1}(A/I)$.
\end{theorem} 

\noindent
\textit{Proof.} On the left, the relation is in $ (A/I) \times T $: $ ([a], [s]) \equiv ([a'], [s']) $ iff (ring notation) $([a][s'] - [a'][s]) [s''] = [0]$ iff $[as's'' - a'ss''] = [0]$. On the right, the relation works in $ (A/I) \times S $: $ ([a], s) \equiv ([a'], s') $ iff (module notation) $s'' (s'[a] - s[a']) = [0]$ iff $[as's'' - a'ss''] = [0]$. The conditions are identical so the classes must be in bijective. However, they are not identical as sets, so saying \textit{equals} is too much.
\qed

\bigskip
\begin{theorem}
What is $S^{-1}\textgoth{a}$?
\end{theorem}

\noindent
It can be either an $S^{-1}A$-module $S^{-1}\textgoth{a}$, because $\textgoth{a}$ is an $A$-module, or the extension $S^{-1}\textgoth{a} = \textgoth{a} \: S^{-1}A$ in $S^{-1}A$ of the ideal $\textgoth{a}$ in $A$. In both cases elements of $S^{-1}\textgoth{a}$ are written as $a/s$ with $a \in \textgoth{a}$, $s \in S$, but they come from different sets. In the first, module case, $a/s$ is in the quotient of $\textgoth{a} \times S$, in the second, extension ideal case, $a/s$ is in the quotient of $A \times S$. We are talking of $S^{-1}A$-modules, not rings, so there can only be $A$-module and $S^{-1}A$-module isomorphism:
\[
  \textgoth{a} \times S / \sim_\textgoth{a} \; \; \ni \; a/s \mapsto a/s \; \in \; A \times S / \sim_A
\]
\qed

\bigskip
\begin{theorem}
What is $S^{-1}\textgoth{a}$ in case $S = A \setminus \textgoth{p}$?
\end{theorem}

\noindent
It can be either an $A_\textgoth{p}$-module $\textgoth{a}_\textgoth{p}$, because $\textgoth{a}$ is an $A$-module, or the extension $S^{-1}\textgoth{a} = \textgoth{a} \: A_\textgoth{p}$ in $A_\textgoth{p}$ of the ideal $\textgoth{a}$ in $A$. In both cases elements of $S^{-1}\textgoth{a}$ are written as $a/s$ with $a \in \textgoth{a}$, $s \notin \textgoth{p}$, but they come from different sets. In the first, module case, $a/s$ is in the quotient of $\textgoth{a} \times (A \setminus\textgoth{p})$, in the second, extension ideal case, $a/s$ is in the quotient of $A \times (A \setminus\textgoth{p})$. We are talking of $A_\textgoth{p}$-modules, not rings, so there can only be an $A$-module and $A_\textgoth{p}$-module isomorphism:
\[
  \textgoth{a} \times (A \setminus\textgoth{p}) / \sim_\textgoth{a} \; \; \ni \; a/s \mapsto a/s \; \in \; A \times (A \setminus\textgoth{p}) / \sim_A
\]
\qed

\bigskip
\begin{theorem}
Case $\textgoth{a} = \textgoth{p}$, a prime ideal. What is $S^{-1}\textgoth{p}$?
\end{theorem}

\noindent
It can be either the $A_\textgoth{p}$-module $\textgoth{p}_\textgoth{p}$, because $\textgoth{p}$ is an $A$-module, or the extension $\textgoth{p} A_\textgoth{p}$ in $A_\textgoth{p}$ of the ideal $\textgoth{p}$ in $A$, via the canonical $A \rightarrow A_\textgoth{p} : a \mapsto a/s$. Looks like we don't have the $\cdot_\textgoth{p}$-instead-of-$S^{-1} \cdot$ notation in the ideal extension case, but then, the quotient notation $ A_\textgoth{p} / \textgoth{p}_\textgoth{p}$ is used, which makes sense only if $\textgoth{p}_\textgoth{p}$ is an ideal in $A_\textgoth{p}$
\[
   \textgoth{p}_\textgoth{p} = \textgoth{p} \: A_\textgoth{p}
\]
\qed

\bigskip
\begin{theorem}
The contraction of $S^{-1}\textgoth{p}$ is $\textgoth{p}$.
\end{theorem}

\noindent
\begin{equation*}
\begin{split}
\{a: a/1 \in S^{-1}\textgoth{p}\} & = \{ a: a/1 = a'/s' \; \text{for some} \; a' \in \textgoth{p}, s' \notin \textgoth{p} \} \\
& = \{a: \exists u \in S \; (as' - a')u = 0 \} \\
& = \{a: \exists u \in S, s' \in S, a' \in \textgoth{p} \quad as'u = a'u  \}
\end{split}
\end{equation*}
The set finally is $\textgoth{p}$: \\
$\implies$: $as'u \in \textgoth{p}$, so $a \in \textgoth{p}$ or $s'u \in \textgoth{p}$ but $\textgoth{p} \cap S = \emptyset$; must be $a \in \textgoth{p}$. \\
$\impliedby$: $a \in \textgoth{p}$; $(a \cdot 1 - a)\cdot 1 = 0$, so $a$ is in the set. \\
\qed

\bigskip
\begin{theorem}
What is $\textgoth{a}_\textgoth{p}$?
\end{theorem}

\noindent
It is the $A$-module $\textgoth{a}$ localized at $\textgoth{p}$. It is an $A_\textgoth{p}$-module. We also use this notation for the ideal $S^{-1}\textgoth{a}$ of $S^{-1}A$, where $S = A \setminus \textgoth{p}$. How are they isomorphic? $a/s \mapsto a/s$ with $a \in \textgoth{a}, s \notin $.  Of what it is an isomorphism? Of $A$-modules, of $A_\textgoth{p}$-modules. They are not rings.
\qed

\bigskip
\begin{theorem}
What is $\textgoth{p}_\textgoth{p}$?
\end{theorem}

\noindent
It is the $A$-module $\textgoth{p}$ localized at $\textgoth{p}$. We also use this notation for the ideal $S^{-1}\textgoth{p}$ of $S^{-1}A$, where $S = A \setminus \textgoth{p}$, that is, the ideal $\textgoth{p}A_\textgoth{p}$.
\qed

\bigskip
\begin{theorem}
What is $ \textgoth{p} A_\textgoth{p}$ ?
\end{theorem}

% \noindent
As $A_\textgoth{p}$ is an $A$-module, we can multiply it by a prime ideal in $A$ in a standard way

\[
 \sum a'_i \frac{a_i}{s_i} = \sum \frac{a' a_i}{s_i}
\]

\noindent
After bringing to a common denominator, this is
\[
 \frac{a}{s}
\]
\noindent
with $a \in \textgoth{p}$, so $ \textgoth{p} A_\textgoth{p}$ is the single maximal ideal of the local ring $A_\textgoth{p}$ .
\qed

\bigskip
\begin{theorem}
What are $\textgoth{p}A_\textgoth{p}$ and $\textgoth{a}A_\textgoth{p}$?
\end{theorem}

\noindent
In the Solutions by Y. P. Gaillard, the residue field is $A_\textgoth{p} / \textgoth{p}A_\textgoth{p}$, so the single maximal ideal of $A_\textgoth{p}$ from the Example 1, p. 38 of ItCA must be just $\textgoth{p}A_\textgoth{p}$
\[
  \textgoth{p}A_\textgoth{p} = \{a/s: a \in \textgoth{p}, s \notin \textgoth{p}\}
\]

\noindent
Then $\textgoth{a}A_\textgoth{p}$ must be the generalization
n
\qed

\bigskip
\begin{theorem}
When $S=A \setminus \textgoth{p}$, as $A_\textgoth{p}$-modules
\[
S^{-1}\textgoth{a} = \textgoth{a}A_\textgoth{p} = \textgoth{a}_\textgoth{p} 
\]
\[
S^{-1}\textgoth{p} = \textgoth{p}A_\textgoth{p} = \textgoth{p}_\textgoth{p} 
\]
\[
S^{-1}\textgoth{p'} = \textgoth{p'}A_\textgoth{p} 
\]
for prime ideal $\textgoth{p'} \subseteq \textgoth{p}$ .
\end{theorem}
\qed

\bigskip
\begin{theorem}
Same notation for general $S$
\[
S^{-1}\textgoth{a} = \textgoth{a}S^{-1}A
\]
\end{theorem}
\qed

\bigskip
\begin{theorem}
How is $A_\textgoth{p}$ an $A$-module ?
\end{theorem}

% \noindent
The canonical map $\phi: A \rightarrow A_\textgoth{p}: a \mapsto \frac{a}{1}$ gives the multiplication by scalars from $A$

\[
 a' \frac{a}{s} = \phi(a') \frac{a}{s} = \frac{a'}{1} \frac{a}{s} = \frac{a' a}{s}
\]
\qed

\bigskip
\begin{theorem}
\label{b_q_as_a_q}
How is $B_\textgoth{q}$ an $A_\textgoth{p}$-module?
\end{theorem}

\noindent
Let $g = \psi \circ f$ be the composition $A \rightarrow B \rightarrow T^{-1}B : a \rightarrow f(a) \rightarrow f(a)/1 $. This composition sends $s \in S$ to a unit in $T^{-1}B$, as $(f(s)/1)(1/f(s)) = 1/1$, where $f(s) \in f(S) = f(A \setminus \textgoth{p}) \subseteq B \setminus \textgoth{q} = T$. Why the inclusion? If $a \notin \textgoth{p} = f^{-1}(\textgoth{q})$ then $f(a) \notin \textgoth{q}$. By the universal property of the ring of fractions, $g$ factorizes
\[
     \begin{tikzcd}
     A \arrow{r}{\phi} \arrow{d}{f} \arrow{dr}{g} & S^{-1}A \arrow{d}{h} \\
     B \arrow{r}{\psi} & T^{-1}B
     \end{tikzcd}
\]

\noindent
where the recipe for $h$ is given in \textbf{Proposition 3.1} of ItCA as $a/s \mapsto g(a)g(s)^{-1} = (f(a)/1)(1/f(s)) = f(a)/f(s)$.
\qed

\bigskip
\begin{theorem}
What is $B_\textgoth{p}$ ?
\end{theorem}

\noindent
(For $f: A \mapsto B$ and $\textgoth{p}$ a prime ideal of $A$). \\
The ring $B$ is an $A$-module by the restriction of scalars. We can localize it in the prime ideal $\textgoth{p}$  of $A$. The cartesian product is $B \times (A \setminus \textgoth{p})$, the relation is 
\[
(b, s) \sim (b', s') \iff \exists t \notin \textgoth{p} \; \; t(sb' - s'b) = 0
\]
The condition reads
\[
f(t)(f(s)b' - f(s')b) = 0
\]
The obvious addition 
\[
\frac{b}{s} + \frac{b'}{s'} = \frac{s'b + sb'}{ss'} = \frac{f(s')b + f(s)b'}{ss'}
\]
The obvious scalar multiplication
\[
\frac{a}{s'} \cdot \frac{b}{s} = \frac{ab}{s's} = \frac{f(a)b}{s's}
\]
\qed

\bigskip
\begin{theorem}
How is $B_\textgoth{p}$ an $A_\textgoth{p}$-module?
\end{theorem}

\noindent
And $f$ is an homomorphism of $A$ modules: \[ f(a'a) = f(a')f(a) = a' \cdot f(a) \] This gives rise to an $A_\textgoth{p}$-module homomorphism $S^{-1}f: A_\textgoth{p} \rightarrow B_\textgoth{p}$
\[
a/s \mapsto f(a)/s
\]
See how it is different from the map $A_\textgoth{p} \rightarrow B_\textgoth{q}$
\[
a/s \mapsto f(a)/f(s)
\]
By the restriction of scalars, $B_\textgoth{p}$ is an $A_\textgoth{p}$-module.
\qed

\bigskip
\begin{theorem}
$B_\textgoth{p}$ is a ring.
\end{theorem}

\noindent
The multiplication 
\[
\frac{b}{s} \cdot \frac{b'}{s'} = \frac{bb'}{ss'}
\]
is distributive over the addition.
\begin{equation*}
\begin{split}
\frac{b''}{s''} \left( \frac{b'}{s'} + \frac{b}{s} \right)
& = \frac{b''}{s''}\frac{sb' + s'b}{s's} \\
& = \frac{b''(sb' + s'b)}{s''s's'} \\
& = \frac{sb''b' + s'b''b}{s''s's} \\
& = \frac{b''b'}{s''s'} + \frac{b''b}{s''s}
\end{split}
\end{equation*}
\begin{equation*}
\begin{split}
\frac{b''}{s''} \frac{b'}{s'} + \frac{b''}{s''} \frac{b}{s}
& = \frac{s''sb''b' + s''s'b''b}{s''s's''s} \\
& = \frac{f(s'')f(s)b''b' + f(s'')f(s')b''b}{s''s's''s} \\
& = \frac{f(s'')f(s)b''b'}{s''s's''s} + \frac{f(s'')f(s')b''b}{s''s's''s}
\end{split}
\end{equation*}
How can we cancel here? In a general $S^{-1}A$-module $S^{-1}M$
\[
\frac{f(s)m}{s} = \frac{s \cdot m}{s} = \frac{s}{s} \cdot \frac{m}{1} = \frac{1}{1} \cdot \frac{m}{1} = \frac{m}{1}
\]
With this cancellation rule, both sides of the distributivity become equal.
\qed

\bigskip
\begin{theorem}
$S^{-1}B$ (an $S^{-1}A$-module)  is a ring.
\end{theorem}

\noindent
By argument identical to that for the $B_\textgoth{p}$ ring.
\qed

\bigskip
\begin{theorem}
A ring homomorphism $S^{-1}A \rightarrow S^{-1}B$.
\end{theorem}

\noindent
It is 
\[
\frac{a}{s} \mapsto \frac{f(a)}{s}
\]
Preservation of the multiplication is immediately verified.
\qed

\bigskip
\begin{theorem}
A ring homomorphis $A_\textgoth{p} \rightarrow B_\textgoth{p}$.
\end{theorem}

\noindent
It is 
\[
\frac{a}{s} \mapsto \frac{f(a)}{s}
\]
Preservation of the multiplication is immediately verified.
\qed

\bigskip
\begin{theorem}
The ring $f(S)^{-1}B$.
\end{theorem}

\noindent
The subset $f(S)$ of the ring $B$ is multiplicatively closed, and we can take the ring of fractions. The construction starts from $B \times f(S)$,
\[
(b, f(s)) \sim (b', f(s')) \iff \exists u \in S \; f(u)(f(s')b - f(s)b') = 0
\]
\qed

\bigskip
\begin{theorem}
The rings $S^{-1}B$ and $f(S)^{-1}B$ are isomorphic via $b/s \mapsto b/f(s)$
\end{theorem}

\noindent
The well-definition and injectivity are easily verified and the surjectivity is obvious.
\qed

\bigskip
\begin{theorem}
The homomorphisms $B \rightarrow S^{-1}B: b \mapsto b/s$ and $B \rightarrow f(S)^{-1}B: b \mapsto b/f(s)$.
\end{theorem}
8
\noindent
The second is natural as $f(S)$ is a multiplicatively closed subset of $B$. The first can arise from the isomorphism of both rings, making the diagram
\[
\begin{tikzcd}
  & S^{-1}B \arrow[dd, "\cong"] \\
  B \arrow[ur] \arrow[dr] \\
  & f(S)^{-1}B
\end{tikzcd}
\]
\[
\begin{tikzcd}
  & b/1 \arrow[dd, mapsto] \\
  b \arrow[ur] \arrow[dr, mapsto] \\
  & b/f(1) = b/1
\end{tikzcd}
\]
commutative, or can be verified directly, and the diagram after it. The top homomorphism 
\qed

\bigskip
\begin{theorem}
What are general ideals of $f(S)^{-1}B$ and $S^{-1}B$?
\end{theorem}

\noindent
Every ideal of $f(S)^{-1}B$ is an extended ideal $f(S)^{-1}\textgoth{b} = \textgoth{b}B_\textgoth{p} = \{b/f(s): b\in \textgoth{b}, s \in S\}$. The isomorphic set in $S^{-1}B$ is $\textgoth{b}S^{-1}B = \{b/s: b\in \textgoth{b}, s \in S\}$.
\qed

\bigskip
\begin{theorem}
What are general prime ideals of $f(S)^{-1}B$ and $S^{-1}B$?
\end{theorem}

\noindent
Prime ideals of $f(S)^{-1}B$ are in 1-1 correspondence with prime ideals of $B$ not meeting $f(S)$.
\[
  \textgoth{q} \longleftrightarrow f(S)^{-1}\textgoth{q} = \textgoth{q}f(S)^{-1}B
\]
Contraction of the right on the left, extension of the left on the right.
\qed

\bigskip
\begin{theorem}
What are general prime ideals of $B_\textgoth{p}$ ?
\end{theorem}

\noindent
$B_\textgoth{p} = S^{-1}B \cong f(S)^{-1}B$ for $S = A \setminus \textgoth{p}$ . Prime ideals of $f(A \setminus \textgoth{p})^{-1}B$ are in 1-1 correspondence with prime ideals of $B$ not meeting $f(A \setminus \textgoth{p})$. We have no better option than using $f(A \setminus \textgoth{p})$ here.
\[
  \textgoth{q} \longleftrightarrow f(A \setminus \textgoth{p})^{-1}\textgoth{q} = \textgoth{q}B_\textgoth{p}
\]
Contraction of the right on the left, extension of the left on the right.
\qed

\bigskip
\begin{theorem}
How does $(S^{-1}f)^*: \textnormal{Spec}(S^{-1}B) \rightarrow \text{Spec}(S^{-1}B)$ work? 
\end{theorem}

\noindent
We show that $(S^{-1}f)^*(S^{-1}\textgoth{q}) = \textgoth{p}$ where $\textgoth{p} = f^{-1}(\textgoth{q})$.
\begin{equation*}
\begin{split}
  (S^{-1}f)^*(S^{-1}\textgoth{q}) 
    & = (S^{-1}f)^*(\{ b/s: b \in \textgoth{q}, s \in S\}) \\
    & = \{ a/s: (S^{-1}f)(a/s) \in S^{-1}\textgoth{b} \} \\
    & = \{ a/s: f(a)/s \in \textgoth{b} \}
\end{split}
\end{equation*}

We can show that this is $S^{-1}\textgoth{p}$. \\

The $\subseteq$: $f(a)/s = b/s'$ for some $b \in \textgoth{q}, s \in S$; we move to $f(S)^{-1}B$; $f(a)/f(s) = b/f(s')$; $(f(a)f(s') - bf(s))f(u) = 0$ for some $u \in S$; $f(a)f(s')f(u) = bf(s)f(u) \in \textgoth{b}$; $f(a) \in \textgoth{q}$; $a \in f^{-1}(\textgoth{q}) = \textgoth{p}$; $a/s \in S^{-1}\textgoth{p}$. 

What if we did not move to $f(S)^{-1}B$? $u(s'f(a) - sb) = 0$ in the $S^{-1}A$-module $S^{-1}B$; $us'f(a) = usb$. But what is the multiplication by scalar from $A$? It is multiplication by $f$ of it. $f(u)f(s')f(a) = f(u)f(s)b \in \textgoth{q}$ ... We proceed the same way.

The $\supseteq$: $a/s \in S^{-1}\textgoth{p}$; $a \in \textgoth{p}$; $f(a) \in \textgoth{q}$; $f(a)/s \in S^{-1}\textgoth{q}$.

Now we know that $(S^{-1}f)^*oll$ is the restriction of $f^*$ to $\psi^*(\textrm{Spec}(S^{-1}B)) = S^{-1}Y$
\qed

\bigskip
\begin{theorem}
How is $B_\textgoth{q} / \textgoth{q}_\textgoth{q}$ = $B_\textgoth{q}/\textgoth{q}B_\textgoth{q}$ an $A_\textgoth{p} / \textgoth{p}_\textgoth{p}$-module?
\end{theorem}

\noindent
We know the map $A_\textgoth{p} \rightarrow B_\textgoth{q} : a/s \mapsto f(a)/f(s)$ from \ref{b_q_as_a_q}.
The kernel of the composition $A_\textgoth{p} \rightarrow B_\textgoth{q} \rightarrow B_\textgoth{q} / \textgoth{q} B_\textgoth{q}$ : $a/s \mapsto f(a)/f(s) + \textgoth{q} B_\textgoth{q}$ contains $\textgoth{p} A_\textgoth{p}$: 
element of $\textgoth{p}A_\textgoth{p}$ is $a/s$ where $a \in \textgoth{p}, s \notin \textgoth{p}$ ; it follow that $f(s) \notin \textgoth{q}$ (otherwise $s \in f^{-1}(\textgoth{q}) = \textgoth{p}$); so the image in the first map of $a/s$ is in $\textgoth{q}B_\textgoth{q}$, the kernel of the second map, then $a/s$ is in the kernel of the composition.
The composition then factors through $A_\textgoth{p} / \textgoth{p} A_\textgoth{p} \rightarrow B_\textgoth{q} / \textgoth{q} B_\textgoth{q}$: $a/s + \textgoth{p} A_\textgoth{p} \mapsto f(a)/f(s) + \textgoth{q} B_\textgoth{q}$. This is a ring homomorphism that makes $B_\textgoth{q} / \textgoth{q}_\textgoth{q}$ an $A_\textgoth{p} / \textgoth{p}_\textgoth{p}$-module.
\qed

\bigskip98
\begin{theorem}
What is $\textgoth{p} M_\textgoth{p}$ ?
\end{theorem}

\noindent
When $M_\textgoth{p}$ is seen as an $A$-module, $\textgoth{p} M_\textgoth{p} = \{am/s: a \in \textgoth{p}, m \in M, s \notin \textgoth{p} \}$. When $M_\textgoth{p}$ is seen as an $A_\textgoth{p}$-module, $\textgoth{p}$ is not even an ideal in $A_\textgoth{p}$, but its extension, $\textgoth{p} A_\textgoth{p}$ is, and $(\textgoth{p} A_\textgoth{p}) M_\textgoth{p} = \{(a/s')(m/s): a \in \textgoth{p}, m \in M, s, s' \notin \textgoth{p} \} = \{am/s: a \in \textgoth{p}, m \in M, s \notin \textgoth{p} \}$, the same set, which we write $\textgoth{p} M_\textgoth{p}$ for:

\[
  \textgoth{p} M_\textgoth{p} = (\textgoth{p} A_\textgoth{p}) M_\textgoth{p}
\]
\qed

\bigskip
\begin{theorem}
What is $\textgoth{p} B$?
\end{theorem}

% \noindent
For $f: A \rightarrow B$, we can think in two ways. As we identify $ab = f(a)b$,  $\textgoth{p} B = \{ \sum a_i b_i = \sum f(a_i)b_i : a_i \in \textgoth{p}, b_i \in B \}$ is the extension $f(\textgoth{p}) B$ of the ideal $\textgoth{p}$. The second way is that $B$ is an $A$-module, and $\textgoth{p}$ a prime ideal in $A$, so we can form $ \textgoth{p} B = \{ \sum a_i b_i = \sum f(a_i)b_i \} $ with $a_i \in \textgoth{p}$, $b_i \in B$, getting the same set.

\qed

\bigskip
\begin{theorem}
What is $ \textgoth{p} B_\textgoth{p}$?
\end{theorem}

% \noindent
$B_\textgoth{p}$ is an $A$ - module, $\textgoth{p}$ is a prime ideal of $A$, so $ \textgoth{p} B_\textgoth{p}$ makes sense and consists of finite sums $\sum a_i (b_i / s_i) = \sum (a_i b_i)/s_i$ where $a_i \in \textgoth{p}$, $b_i \in B$, and $s_i \in A \setminus \textgoth{p}$.
After bringing to common denominator, the sum becomes $ab/s$ where $a \in \textgoth{p}$, $b \in B$ and $s \in A \setminus \textgoth{p}$. We observe that $b \in \textgoth{p} B$+.
\qed

\bigskip
\begin{theorem}
What is $(\textgoth{p} A_\textgoth{p}) B_\textgoth{p}$ ?
\end{theorem}

% \noindent
As $B_\textgoth{p}$ is an $A_\textgoth{p}$-module, and $\textgoth{p} A_\textgoth{p}$ is the single maximal ideal of the local ring $A_\textgoth{p}$, Any element is, from the definition of the ideal-by-module and from the general element of $\textgoth{p} A_\textgoth{p}$ ($a \in \textgoth{p}$)

\[
 \sum_i \frac{a_i}{s'_i} \frac{b_i}{s_i} = \sum \frac{ab}{s's}
\]

\noindent
After bringing to a common denominator, this becomes
\[
 ab/s = f(a)b/s
\]

\noindent
where $a \in \textgoth{p}$. Notice we got the general element of $\textgoth{p} B_\textgoth{p}$, so

\[
 (\textgoth{p} A_\textgoth{p}) B_\textgoth{p} = \textgoth{p} B_\textgoth{p}
\]
\qed

\bigskip
\begin{theorem}
The extension in $B_\textgoth{p}$ of the maximal ideal $\textgoth{p}A_\textgoth{p}$ of $A_\textgoth{p}$ is $\textgoth{p}B_\textgoth{p}$
\end{theorem}

\noindent
\begin{align*}
  B_\textgoth{p}(S^{-1}f)(\textgoth{p}A_\textgoth{p}) & = B_\textgoth{p}(S^{-1}f) \bigg \{\frac{a}{s}: a \in \textgoth{p}, s \notin \textgoth{p} \bigg \} \\
  & = B_\textgoth{p} \bigg\{ \frac{f(a)}{s}: a \in \textgoth{p}, s \notin \textgoth{p} \bigg\} \\
  & = \bigg\{ \frac{bf(a)}{s}: a \in \textgoth{p}, b \in B, s \notin \textgoth{p} \bigg\} \\
  & = \bigg\{ \frac{ab}{s}: a \in \textgoth{p}, b \in B, s \notin \textgoth{p} \bigg\} 
\end{align*}
We know from Facts that this is $\textgoth{p}B_\textgoth{p}$.
\qed

\bigskip
\begin{theorem}
How
\[
\frac{(B \otimes_A M)_\textgoth{q}}{\textgoth{q} (B \otimes_A M)_\textgoth{q}} 
\cong
\frac{B_\textgoth{q}}{\textgoth{q}_\textgoth{q}} \otimes_B B \otimes_A M
\]
?
\end{theorem}

\noindent
Proposition 3.5 states, in the language of subscript-$\textgoth{p}$, that $M_\textgoth{p} \cong A_\textgoth{p} \otimes_A M$ over $A_\textgoth{p}$. Here $(B \otimes_
A M)_\textgoth{q} \cong B_\textgoth{q} \otimes_B (B \otimes_A M)$ over $B_\textgoth{p-}$. Then

\begin{gather*}
\frac{B_\textgoth{q} \otimes_B (B \otimes_A M)}{(\textgoth{q} B_\textgoth{q})(B_\textgoth{q} \otimes_B (B \otimes_A M))}
\cong
\frac{B_\textgoth{q}}{\textgoth{q} B_\textgoth{q}} \otimes_{B_\textgoth{q}} (B_\textgoth{q} \otimes_B (B \otimes_A M)) \\
\cong
\frac{B_\textgoth{q}}{\textgoth{q}_\textgoth{q}} \otimes_B B \otimes_A M
\end{gather*}
\qed

\noindent
The first equality is from Exercise 2.2: $M/\textgoth{a}M \cong 
A/\textgoth{a} \otimes_A M$. 
In P. Y. Gaillard solution to ItCA Exercise 3.19 (viii).

\bigskip
\begin{theorem}
How
\[
\frac{A_\textgoth{p}}{\textgoth{p} A_\textgoth{p}} \otimes_A M 
\cong
\frac{M_\textgoth{p}}{\textgoth{p} M_\textgoth{p}}
\]
?
\end{theorem}

\begin{gather*}
\frac{A_\textgoth{p}}{\textgoth{p} A_\textgoth{p}} \otimes_A M 
\cong 
\frac{A_\textgoth{p}}{\textgoth{p} A_\textgoth{p}} \otimes_{A_\textgoth{p}} A=_\textgoth{p} \otimes_A M 
\cong
\frac{A_\textgoth{p}}{\textgoth{p} A_\textgoth{p}} \otimes_{A_\textgoth{p}} M_\textgoth{p} 
\cong
\frac{M_\textgoth{p}}{\textgoth{p} M_\textgoth{p}}
\end{gather*}
\noindent
The second by Proposition 3.5, the third by Exercise 2.2. \\
In Y. P. Gaillard solution of ItCA Exercise 3.19 (viii).
\qed

\bigskip
\begin{theorem}
How $(B \otimes_A M)_\textgoth{q} = B_\textgoth{q} \otimes_{A_\textgoth{p}} M_\textgoth{p}$ ?
\end{theorem}

\noindent
\begin{equation*}
\begin{split}
(B \otimes_A M)_\textgoth{q} & = B_\textgoth{q} \otimes_B (B \otimes_A M) \\
                             & = B_\textgoth{q} \otimes_A M \\
                             & = (B_\textgoth{q} \otimes_{A_\textgoth{p}} A_\textgoth{p}) \otimes_A M \\
                             & = B_\textgoth{q} \otimes_{A_\textgoth{p}} (A_\textgoth{p} \otimes_A M) \\
                             & = B_\textgoth{q} \otimes_{A_\textgoth{p}} M_\textgoth{p}
\end{split}
\end{equation*}
\noindent
The first and the last equalities are applications of Proposition 3.5:
\[
  S^{-1}A \otimes_A M \cong S^{-1}M
\]
\[
  A_\textgoth{p} \otimes_A M \cong M_\textgoth{p}
\]
\[
\frac{a}{s} \otimes m \mapsto \frac{am}{s}
\]
In Y. P. Gaillard solution to ItCA Exercise 3.19 (iii).
\qed

\bigskip
\begin{theorem}
The diagram
\[
     \begin{tikzcd}
     A_\textgoth{p} \arrow{r}{\phi} \arrow{d}{f} \arrow{dr}{\eta} & A_\textgoth{p}/\textgoth{p}A_\textgoth{p} \arrow{d}{h} \\
     B_\textgoth{q} \arrow{r}{\psi} & B_\textgoth{q}/\textgoth{q}B_\textgoth{q}
     \end{tikzcd}
\]

\[
     \begin{tikzcd}
     a/s \arrow[r, mapsto, "\phi"] \arrow[d, mapsto, "f"] \arrow[dr, mapsto, "\eta"] & a/s + \textgoth{p}A_\textgoth{p} \arrow[d, mapsto, "h"] \\
     f(a)/f(s) \arrow[r, mapsto, "\psi"] & f(a)/f(s) + \textgoth{q}B_\textgoth{q}
     \end{tikzcd}
\]
is commutative.
\end{theorem}

% \noindent
All calculated on the diagram. \qed

Now $\kappa_\textgoth{q} = B_\textgoth{q}/\textgoth{q}B_\textgoth{q}$ is an $A_\textgoth{p}$-module by $A_\textgoth{p} \rightarrow A_\textgoth{p}/\textgoth{p}A_\textgoth{p} \rightarrow B_\textgoth{q}/\textgoth{q}B_\textgoth{q}$ (with the formula as on the bottom diagram) and we may tensor over $A_\textgoth{p}$. 

If a field $K$ is an $A$-module for some ring $A$, can it be a zero $A$-module?
\[
 1_A 1_K = 1_k \ne 0_K
\]
It cannot.

Now that $\kappa_\textgoth{q} \otimes_{A_\textgoth{p}} M_\textgoth{p}/\textgoth{p}M_\textgoth{p} = 0$, both tensorands finitely generated, and $\kappa_\textgoth{q} \ne 0$, it must be $M_\textgoth{p}/\textgoth{p}M_\textgoth{p} = 0$ by ItCA Exercise 2.3. 

In solution of ItCA 3.19 (viii) by J. D. Taylor.

\bigskip
\begin{theorem}
How $B_\textgoth{p}/\textgoth{p} B_\textgoth{p} = A_\textgoth{p}/\textgoth{p} A_\textgoth{p} \otimes_{A_\textgoth{p}} B_\textgoth{p}$ ?
\end{theorem}

% \noindent
Apply Exercise 2.2
\[
 A/\textgoth{a} \otimes_A M \cong M / \textgoth{a} M 
\]

\noindent
to $M \coloneqq B_\textgoth{p}$, $A \coloneqq A_\textgoth{p}$, $\textgoth{a} \coloneqq \textgoth{p} A_\textgoth{p}$

\[
 A_\textgoth{p}/\textgoth{p} A_\textgoth{p} \otimes_{A_\textgoth{p}} B_\textgoth{p} = B_\textgoth{p}/(\textgoth{p} A_\textgoth{p}) B_\textgoth{p}
\]

\noindent
now apply $(\textgoth{p} A_\textgoth{p}) B_\textgoth{p} = \textgoth{p} B_\textgoth{p}$ . 
\qed

\vspace{1.5em}
\begin{theorem}
How $A_\textgoth{p} \otimes_A B \cong B_\textgoth{p}$ ?
\end{theorem}

% \noindent
Apply Proposition 3.5: $S^{-1}A \otimes_A M \cong S^{-1}M$ . 
\qed

\vspace{1em}
We now understand the isomorphisms in the solution of ItCA's 3.21(iv) by J D. Taylor.

\begin{equation*}
\begin{split}
 B_\textgoth{p}/\textgoth{p}B_\textgoth{p} 
 & = A_\textgoth{p}/\textgoth{p}A_\textgoth{p} 
 \otimes_{A_\textgoth{p}} B_\textgoth{p} \\ 
 & = K_\textgoth{p} \otimes_{A_\textgoth{p}} A_\textgoth{p} \otimes_A B \\
 & = K_\textgoth{p} \otimes_A B
\end{split}
\end{equation*}

\bigskip
\begin{theorem}
$\textgoth{p} \supseteq \textgoth{a} \iff S^{-1}\textgoth{p} \supseteq S^{-1}\textgoth{a}$
\end{theorem}

\noindent
The $\implies$ direction is universal for ideal extensions. For the $\impliedby$ direction, 
$(S^{-1}\textgoth{p})^c \supseteq (S^{-1}\textgoth{a})^c$ meaning $\textgoth{p} \supseteq \textgoth{a}^{ec} \supseteq \textgoth{a}$
\qed

\vspace{1.5em}
\begin{theorem}
If $\textgoth{p} \supseteq \textgoth{a}$ then $\textgoth{p} \supseteq \textgoth{a}^{ec} \supseteq \bigcup\limits_{s \in S} (\textgoth{a}:s)$
\end{theorem}

\noindent
If $x \in (\textgoth{a}:s)$ then $xs \in \textgoth{a} \subseteq \textgoth{p}$ then $xs \in \textgoth{p}$ then $x \in \textgoth{p}$ or $s \in \textgoth{p}$ but $\textgoth{p} \cap S = O$ so $x \in \textgoth{p}$. 
\qed

\begin{theorem}
$S^{-1}(\aaa M) = S^{-1}\aaa S^{-1}M = \aaa S^{-1}M$
\end{theorem}

What is $S^{-1}(\aaa M)$ ? $\aaa M$ is a submodule of the $A$-module $M$ that is, it is an $A$-module. $S^{-1}(\aaa M)$. $S^{-1}(\aaa M)$ is the module of fractions, with respect to $S$. Its construction starts from $\aaa M \times S$. Its elements are $am/s$, with $a \in \aaa$, classes in the quotient of $\aaa M \times S$.

What is $S^{-1}\aaa \cdot S^{-1}M$ ? $S^{-1}\aaa$ is the extension of $\aaa$ in $S^{-1}A$. Its elements are $a/s$ with $a \in \aaa$. $S^{-1}M$ is the module of fractions of $M$ with respect to $S$. Its elements are $m/s$. It is an $S^{-1}A$-module so we can multiply it by the ideal $S^{-1}\aaa$ of $S^{-1}A$. The elements of $S^{-1}\aaa \cdot S^{-1}M$ are $(a/s)(m/s')$ where $a \in \aaa$. Any element can be written as $am/s$ with $a \in \aaa$. But the construction of $S^{-1}M$ started from $M \times S$. Any $am/s$ is a class in the quotient of $M \times S$.

What is $\aaa \cdot S^{-1}M$ ? $S^{-1}M$ is an $S^{-1}A$-module, but $\aaa$ is an ideal in $A$, not $S^{-1}A$. Still $S^{-1}M$ is also an $A$-module through the restriction of scalars
\[
A \overset{\phi}\rightarrow S^{-1}A
\]
\[
a \mapsto a/1
\]
The scaling by an element of $A$ is
\[
a \cdot \frac{m}{s} = \phi(a) \cdot \frac{m}{s} = \frac{a}{1}\frac{m}{s} = \frac{am}{s}
\]
Now $\aaa \cdot S^{-1}M$ are $am/s$ with $a \in \aaa$. Any of them is a class in the quotient of $M \times S$.

\begin{theorem}
Case $S = A\setminus\ppp$
\[
(\aaa M)_\ppp = \aaa_\ppp M_\ppp = \aaa M_\ppp
\]
\end{theorem}
\qed

\begin{theorem}
Case $\aaa = \ppp$
\[
(\ppp M)_\ppp = \ppp_\ppp M_\ppp = \ppp M_\ppp
\]
\end{theorem}
\qed

\begin{theorem}
Case $M = A$
\[
\ppp_\ppp = \ppp_\ppp A_\ppp = \ppp A_\ppp
\]
\end{theorem}
\qed

\begin{theorem}
For all $A$-linear map $g: M \rightarrow N$ from $M$ to an $S^{-1}$-module $N$ such that $sm = 0$ for some $s \in S$ and some $m \in M$ implies $g(m) = 0$, there is a unique $S^{-1}A$-linear map $h: S^{-1}M \rightarrow N$ such that $g = h \circ f$:
\[
     \begin{tikzcd}
     M \arrow{r}{g} \arrow{d}{f} & N \\
     S^{-1}M \arrow[dashrightarrow]{ur}{h}
     \end{tikzcd}
\]
\end{theorem}
\noindent
What if $sm = 0$ but $g(m) \neq 0$?
\[
g(sm) = g(0) = 0 
\]
but 
\[
sg(m) = \frac{s}{1}g(m)
\]
(restriction of scalars!) is a nonzero vector scaled by a unit, which cannot be zero. The map becomes non-$A$-linear.

\textit{Existence}. Let $h(m/s) = (1/s)g(m) = s^{-1}g(m)$. Then $h$ will clearly be an $A$-module isomorphism provided it is well-defined. Suppose that $m/s = m'/s'$; then there exists $t \in S$ such that $t(s'm - sm') = 0$; taking $g$ on this, $t(s'g(m) - sg(m')) = 0$; multiplying by $1/tss'$, we get $(1/s')g(m) = (1/s)g(m')$. Thus $h$ is well-defined and we get the existence proved.

\textit{Uniqueness} If $h$ satisfies the condition then $h(m/1) = g(m)/1$ for all $m \in M$; hence, if $s \in S$, $h(m/s) = h((1/s)(m/s)) = (1/s)h(m/s) = (1/s)(g(m)/1) = g(m)/s$ so that $h$ is uniquely determined by $g$.
\qed

\vspace{8mm}

This is the first time we encounter a module over a localized ring that is not itself a localization (in the same multipliset). But maybe it is? 
By the restriction of scalars, $N$ is also an $A$-module with scaling
\[
an = \phi(a)n = \frac{a}{1}n
\]
What if we localize it at $S$? In $N \times S$, the relation is 
\[
(n, s) \sim (n', s') \iff \exists u \in S \; u(s'n - sn') = 0
\]
Under the quantifier there is
\[
\phi(u)(\phi(s')n - \phi(s)n') = 0 
\]
\[
\frac{u}{1}(\frac{s'}{1}n - \frac{s}{1}n') = 0
\]
We can use the universal property after verification of $A$-linearity of the horizontal map $g$. On the left, $N$ is an $A$-module, on the right, it is an $S^{-1}A$-module.
\[
\begin{tikzcd}[column sep=large]
N \arrow{r}{g} \arrow{d}[swap]{f} & N \\
S^{-1}N \arrow{ur}[swap]{h}
\end{tikzcd}
%
\qquad
%
\begin{tikzcd}[column sep=large]
n \arrow[d, mapsto, swap, "f"] \arrow[r, mapsto, "g"] & n \\
n/1\arrow[ur, mapsto, swap, "h"] 
\end{tikzcd}
\]
If $sn = 0$ then in the restriction of scalars this means $(1/s)n = 0$ and since $1/s$ is a unit, $n = 0$, then $g(n) = 0$. Now we can use the universal property. The map $h$ on general element is
\[
n/s \mapsto \frac{1}{s}n
\]
We verify injectivity. Let
\[
\frac{1}{s}n = \frac{1}{s'}n'
\]
Multiplying by $ss'/1$
\[
\frac{s'}{1}n = \frac{s}{1}n'
\]
\[
s'n = sn'
\]
in the $A$-module $N$. Then, in $S^{-1}N$,
\[
\frac{n}{s} = \frac{n'}{s'}
\]
and we have the injectivity proved. Clearly the map is surjective. Then it is an isomorphism
\[
S^{-1}N \cong N
\]
of $S^{-1}A$-modules. 

\begin{theorem}
Any module over a ring of fractions with respect to a multipliset, is a module of fractions with respect to this multipliset. 
\end{theorem}
\qed

\begin{theorem}
$M_\ppp/\ppp M_\ppp \cong (M/\ppp M)_\ppp$
\end{theorem}
We start from the exact sequence
\[
0 \rightarrow \ppp M \rightarrow M \rightarrow M/ \ppp M \rightarrow 0
\]
By the exactness of $S^{-1}$ (Proposition 3.3 of the Book), the sequence
\[
0 \rightarrow (\ppp M)_\ppp \rightarrow M_\ppp \rightarrow (M/\ppp M)_\ppp \rightarrow 0
\]
is exact. As $(\ppp M)_\ppp = \ppp M_\ppp$, the sequence
\[
0 \rightarrow \ppp M_\ppp \rightarrow M_\ppp \rightarrow (M/\ppp M)_\ppp \rightarrow 0
\]
is exact, with ordinary inclusion on the second left.
\qed

\vspace{8mm}

\begin{theorem}
Exercise 2.2 of the Book
\[
A/\aaa \otimes_A M \cong M/\aaa M
\]
on elements.
\end{theorem}
The sequence 
\[
0 \rightarrow \aaa \rightarrow A \rightarrow A/\aaa \rightarrow 0
\]
\[
a \mapsto a \mapsto a + \aaa
\]
is exact. We tensor it with $M$ over $A$
\begin{alignat*}{6}
\aaa \otimes_A M 
  & \rightarrow 
  & A \otimes_A M 
  & \rightarrow 
  & A/\aaa \otimes M     
  & \rightarrow 
  & 0 \\
a \otimes m      
  & \mapsto     
  & a \otimes m   
  & \mapsto     
  & (a + \aaa) \otimes m 
  & \mapsto 
  & 0
\end{alignat*}
By the isomorphism
\begin{align*}
A \otimes_A M & \cong M \\
a \otimes m & \mapsto am \\
1 \otimes m & \mapsfrom m
\end{align*}
\begin{alignat*}{6}
\aaa \otimes_A M 
  & \rightarrow 
  & M 
  & \rightarrow 
  & A/\aaa \otimes M     
  & \rightarrow 
  & 0 \\
a \otimes m      
  & \mapsto     
  & am   
  & \mapsto     
  & (1 + \aaa) \otimes am 
  & \mapsto 
  & 0
\end{alignat*}
The image of the first homomorphism is precisely $\aaa M$, and we take quotient.
\begin{align*}
M/\aaa M & \cong A/\aaa \otimes_A M \\
\overline{m} & \mapsto \overline{1} \otimes m \\
\overline{am} & \mapsfrom \overline{a} \otimes m
\end{align*}
The inverse map has yet to be defined. Consider the map
\begin{align*}
A/\aaa \times M & \longrightarrow M/\aaa M \\
(\overline{a}, m) & \longmapsto \overline{am}
\end{align*}
It is well-defined: if $\overline{a} = \overline{a'}$ then $a - a' \in \aaa$ then $(a - a')m \in \aaa M$ then $am - a'm \in \aaa M$ then $\overline{am} = \overline{a'm}$. It is clearly $A$-bilinear. Then it factors through $A/\aaa \otimes_A M$
\[
\begin{tikzcd}[column sep=large]
A/\aaa \times M \arrow{r}{\otimes} \arrow{rd} & A/\aaa \otimes_A M \arrow{d} \\
& M/\aaa M
\end{tikzcd}
%
\qquad
%
\begin{tikzcd}[column sep=large]
(\overline{a}, m) \arrow[r, mapsto, "\otimes"] \arrow{rd} & \overline{a} \otimes m \arrow[d, mapsto] \\
& \overline{am}
\end{tikzcd}
\]


\qed


\section{Saturated}

\begin{theorem}
For saturated $S$, if $f(a)$ is a unit in $S^{-1}A$, then $a \in S$.
\end{theorem}

\noindent
\textit{Proof.}
\[
  \frac{a}{1} \cdot \frac{b}{t} = \frac{1}{1} 
\]

\[
  \frac{ab}{t} = \frac{1}{1}
\]

\[
   (ab, t) \equiv (1, 1)
\]
 
\[
   (ab - t)u = 0
\]

\[
   abu = tu
\]

\[
   abu \in S
\]

As \( S \) is saturated, \( a \in S \).  \qed

\end{document}S
