\documentclass{article}
\usepackage[utf8]{inputenc}
\usepackage{natbib}
\usepackage{graphicx}
\usepackage{tikz-cd}
\usepackage{tikz}
\usetikzlibrary{cd}
\usepackage{amsmath}
\usepackage{amssymb}
\usepackage{yfonts}
\usepackage{hyperref}
\usepackage{amsthm}
\usepackage{stmaryrd}
\usepackage{mathtools}
% \usepackage{floatrow}
\usepackage{subcaption}

\usepackage{geometry}
\geometry{
 a4paper,
 total={170mm,257mm},
 left=35mm,
 right=35mm,
 top=20mm,
 }

\hypersetup{
    colorlinks=true,
    linkcolor=blue,
    filecolor=magenta,      
    urlcolor=cyan,
}

\urlstyle{same}
%\newcommand{\goth}[1]{\text{\textgoth{#1}}

\newtheorem{theorem}{Step}

\title{Isomorphisms 1 to 2 in Claim 1}

\begin{document}

\maketitle

\textbf{Fact}. The isomorphism 1 to 2.w
\[
  A_\textgoth{p} \otimes_A B/\textgoth{p}B \cong (B/\textgoth{p}B)_\textgoth{p}
\]

\begin{theorem}
How is $A_\textgoth{p} \otimes B/\textgoth{p}B$ a $B$-algebra?
\end{theorem}


We need a ring structure and a homomorphism from $B$. The first is defined in [ItCA] TENSOR PRODUCT OF ALGEBRAS pp 30-31.
\[
    \left( \frac{a}{s} \otimes (b + \textgoth{p}B) \right) \left( \frac{a'}{s'} \otimes (b' + \textgoth{p}B) \right) = \frac{aa'}{ss'} \otimes (bb' + \textgoth{p}B)
\]
but first the $B/\textgoth{p}B$has to be an algebra. That is, to be a ring. We have this in Facts, $\textgoth{p}B = \textgoth{p}^e$, an ideal of $B$ that does not have to be prime, still, $B/\textgoth{p}B$ is a ring.

The $B/\textgoth{p}B$ is a ring and a $B$-module. How is it a $B$-algebra? A homomorphism of rings? One proposition
\[
B \longrightarrow B/\textgoth{p}B
\]
\[
b \mapsto b + \textgoth{p}
\]

In the tensor product of algebras, $A \xrightarrow{f} B$, $A \xrightarrow{g} C$, the mapping from $A$ is $a \mapsto f(a) \otimes g(a)$, here the mapping from $B$ to $A_\textgoth{p} \otimes_A B/\textgoth{p}B$ is
\[
 b \mapsto \frac{b}{1} \otimes (b + \textgoth{p}B
\]
But are they both, the tensorands, $B$-algebras?

\vspace{10px}

What is $(B/\textgoth{p}B)_\textgoth{p}$? \\
$\textgoth{p}B$ is an ideal of $B$ . \\
$B/\textgoth{p}B$ is a ring and an $A$-module. \\
We can take the $(B/\textgoth{p}B)_\textgoth{p}$ $A\textgoth{p}$-module.

What a ring structure do we have on $(B/\textgoth{p}B)_\textgoth{p}$? General element is $(b + \textgoth{p}B)/s$. The obvious proposition is
\[
  \frac{b + \textgoth{p}B}{s} \cdot \frac{b' + \textgoth{p}B}{s'} = \frac{bb' + \textgoth{p}B}{ss'}
\]

Well-definition. Show that if
\[
\frac{b_1 + \textgoth{p}B}{s_1} = \frac{b_2 + \textgoth{p}B}{s_2}, \quad
\frac{b'_1 + \textgoth{p}B}{s'_!} = \frac{b'_2 + \textgoth{p}B}{s'_2} \tag{1}\label{eq:first}
\]
then
\[
\frac{b_1 b'_1 + \textgoth{p}B}{s_1 s'_1} = \frac{b_2 b'_2 + \textgoth{p}B}{s_2 s'_2} 
\tag{2}\label{eq:second}
\]
that is,
\[
v s_2 s_2' b_1 b_1' - v s_1 s_1' b_2 b_2' \in \textgoth{p}B
\tag{3}\label{eq:third}
\]
The \eqref{eq:first} means
\[
u(s_2(b_1 + \textgoth{p}B) - s_1(b_2 + \textgoth{p}B)) = 0, 
\]
\[
u'(s_2'(b_1' + \textgoth{p}B) - s_1'(b_2' + \textgoth{p}B)) = 0
\]
That is,
\[
u s_2 b_1 - u s_1 b_2 \in \textgoth{p}B,
\]
\[
u' s_2' b_1' - u' s_1' b_2' \in \textgoth{p}B,
\]
Then
\[
\overline{u s_2 b_1} = \overline{u s_1 b_2}, \quad 
\overline{u' s_2' b_1'} = \overline{u' s_1' b_2'}
\]
\[
\overline{u s_2 b_1 u' s_2' b_1'} =
\overline{u s_1 b_2 u' s_1' b_2'}
\]
\[
u u' s_2 s_2' b_1 b_1' - u u' s_1 s_1' b_2 b_2' \in  \textgoth{p}B
\]
And it is enough to take $v = u u'$ in \eqref{eq:third}.

Is a localization $B_\textgoth{p}$ of an $A$-algebra $B$ in a prime ideal $\textgoth{p}$ of $A$ a $B$-algebra? It is an $A_\textgoth{p}$-module. As the multiplication we try
\[
\frac{b}{s} \cdot \frac{b'}{s'} = \frac{b b'}{s s'}
\]
Well-definition.
\[
\frac{b_1}{s_1} = \frac{b_2}{s_2}, \quad \frac{b_1'}{s_1'} = \frac{b_2'}{s_2'} 
\implies
\frac{b_1 b_1'}{s_1 s_1'} = \frac{b_2 b_2'}{s_2 s_2'} \ ?
\]
The left
\[
u(s_2 b_1 - s_1 b_2) = 0, \quad u'(s_2' b_1' - s_1' b_2') = 0
\]
\[
u s_2 b_1 = u s_1 b_2, \quad u' s_2' b_1' = u' s_1' b_2'
\]
The right
\[
v(s_2 s_2' b_1 b_1' - s_1 s_1' b_2 b_2') = 0
\]
\[
v s_2 s_2' b_1 b_1' = v s_1 s_1' b_2 b_2'
\]
Take $v = u u'$. \\
Distributivity over addition.
\[
\frac{b}{s} \left( \frac{b'}{s'} + \frac{b''}{s''} \right) = 
\frac{b}{s} \frac{s'' b' + s' b''}{s' s''} = \frac{s'' b b' + s' b b''}{s s' s''}
\]
\[
\frac{b}{s}\frac{b'}{s'} + \frac{b}{s}\frac{b''}{s''} = 
\frac{b b'}{s s'} + \frac{b b''}{s s''} = \frac{s s'' b b' + s s' b b''}{s^2 s' s''} =
\frac{s(s'' b b' + s' b b'')}{s^2 s' s''} = 
\frac{s'' b b'' + s' b b''}{s s' s''}
\]

We now use this fact to get the multiplication in the $A$-algebra $B/\textgoth{p}B$ localized at an ideal $\textgoth{p}$ of $A$, that is, $(B/\textgoth{p}B)_\textgoth{p}$.
\[
\frac{b + \textgoth{p}B}{s} \cdot \frac{b' + \textgoth{p}B}{s'} = 
\frac{(b + \textgoth{p}B)(b' + \textgoth{p}B)}{s s'} = 
\frac{b b' + \textgoth{p}B}{s s'}
\]
which is an easier way to define this multiplication, avoiding lots of calculations.

We now move to to a proposition of mutually reverse maps between 
\[
A_\textgoth{p} \otimes_A B/\textgoth{p}B \quad  \text{and} \quad
(B/\textgoth{p}B)_\textgoth{p}
\]
Both are $B$-algebras. And $B$-algebras' isomorphisms must be commutative with homomorphisms from $B$.

\begin{figure}[htb]
\centering
\begin{subfigure}[T]{0.4\textwidth}
\centering
\begin{tikzcd}
& A_\textgoth{p} \otimes_A B/\textgoth{p}B \arrow[dd]  \\
B \arrow[ur] \arrow[dr] \\
& (B/\textgoth{p}B)_\textgoth{p}
\end{tikzcd}
\end{subfigure}
\centering
\begin{subfigure}[T]{0.4\textwidth}
\centering
\begin{tikzcd}
& 1/1 \otimes (b + \textgoth{p}B) \arrow[dd, mapsto]  \\
b \arrow[ur, mapsto] \arrow[dr] \\
& (b + \textgoth{p}B)/1
\end{tikzcd}
\end{subfigure}
\end{figure}
\noindent
For the top-down, the proposition is
\[
a/s \otimes (b + \textgoth{p}B) \mapsto (ab + \textgoth{p}B)/s
\tag{4}\label{eq:fourth}
\]
For the bottom-up, the proposition is
\[
(b + \textgoth{p}B)/s \mapsto  1/s \otimes (b + \textgoth{p}B)
\tag{5}\label{eq:fifth}
\]

We start with $A$-bilinear map $A_\textgoth{p} \times B/\textgoth{p}B \rightarrow (B/\textgoth{p}B)_\textgoth{p}$
\[
(a/s, b + \textgoth{p}B) \mapsto (ab + \textgoth{p}B)/s
\tag{6}\label{eq:sixth}
\]
First we verify well-defibition then bilinearity.
\[
\frac{a}{s} = \frac{a'}{s'}, \ b + \textgoth{p}B = b' + \textgoth{p}B 
\ \xRightarrow{?} \ 
\frac{ab + \textgoth{p}B}{s} = \frac{a' b' + \textgoth{p}B}{s'}
\]
\[
u(s'a - sa') = 0, \ b - b' \in \textgoth{p}B
\ \xRightarrow{?} \ 
v(s'(ab + \textgoth{p}B) - s(a'b' + \textgoth{p}B)) = 0
\]
\[
us'a = usa', \ b = b' + b'', b'' \in \textgoth{p}B
\ \xRightarrow{?} \ 
vs'ab - vsa'b' \in \textgoth{p}B
\]
By the multiplication property of equality
\[
us'ab = usa'b' + usa'b'',
\]
the second summand being in the ideal $\textgoth{p}B$,
\[
us'ab - usa'b' \in \textgoth{p}B
\]
Taking $v = u$ we see our map well-defined. We now verify the bilinearity. Additivity in the first.
\begin{align*}
\left( \frac{a}{s} + \frac{a'}{s'}, b + \textgoth{p}B \right) 
& = \left( \frac{s'a + sa'}{ss'}, b + \textgoth{p}B \right) \\
& \mapsto \frac{(s'a + sa')b + \textgoth{p}B}{ss'} \\
& = \frac{s'ab + sa'b + \textgoth{p}B}{ss'} \\
& = \frac{s'ab + \textgoth{p}B}{ss'} + \frac{sa'b + \textgoth{p}B}{ss'} \\
& = \frac{s'(ab + \textgoth{p}B)}{ss'} + \frac{s(a'b + \textgoth{p}B)}{ss'} \\
& = \frac{ab + \textgoth{p}B}{s} + \frac{a'b + \textgoth{p}B}{s'} \\
& = \ \mapsfrom \left( \frac{a}{s}, b + \textgoth{p}B \right) + \mapsfrom \left( \frac{a'}{s'}, b + \textgoth{p}B \right)
\end{align*}
Scaling in the first.
\begin{align*}
\left( a' \cdot \frac{a}{s}, b + \textgoth{p}B \right) 
& = \left( \frac{aa'}{s}, b + \textgoth{p}B \right) \\
& \mapsto \frac{a'ab + \textgoth{p}B}{s} \\
& = a' \cdot \frac{ab + \textgoth{p}B}{s} \\
& = a' \cdot \mapsfrom \left( \frac{a}{s}, b + \textgoth{p}B \right)
\end{align*}
Additivity in the second.
\begin{align*}
\left( \frac{a}{s}, b + \textgoth{p}B + b' + \textgoth{p}B \right)
& = \left( \frac{a}{s}, b + b' + \textgoth{p}B \right) \\
& \mapsto \frac{a(b + b') + \textgoth{p}B}{s} \\
& = \frac{ab + ab' + \textgoth{p}B}{s} \\
& = \frac{ab + \textgoth{p}B + ab' + \textgoth{p}B}{s} \\
& = \frac{ab  + \textgoth{p}B}{s} + \frac{ab' + \textgoth{p}B}{s} \\
& = \ \mapsfrom \left( \frac{a}{s}, b + \textgoth{p}B \right) + \mapsfrom \left( \frac{a}{s}, b' + \textgoth{p}B \right)
\end{align*}
Scaling in the second.
\begin{align*}
\left( \frac{a}{s}, a'(b + \textgoth{p}B) \right)
& = \left( \frac{a}{s}, a'b + \textgoth{p}B \right) \\
& \mapsto \frac{aa'b + \textgoth{p}B}{s} \\
& = a' \cdot \frac{ab + \textgoth{p}B}{s} \\
& = a' \cdot \mapsfrom \left( \frac{a}{s}, b + \textgoth{p}B \right)
\end{align*}
The map \eqref{eq:sixth} is now proved bilinear. By the universal property it factors. 
\begin{figure}[htb]
\centering
\begin{subfigure}[T]{0.4\textwidth}
\centering
\begin{tikzcd}
A_\textgoth{p} \times B/\textgoth{p}B \arrow[r] \arrow[rd] & A_\textgoth{p} \otimes_A B/\textgoth{p}B \arrow[d] \\
& (B/\textgoth{p}B)_\textgoth{p}
\end{tikzcd}
\end{subfigure}
\centering
\begin{subfigure}[T]{0.4\textwidth}
\centering
\begin{tikzcd}
(a/s, b + \textgoth{p}B) \arrow[r, mapsto] \arrow[rd, mapsto] & a/s \otimes b + \textgoth{p}B \arrow[d, mapsto] \\
& (ab + \textgoth{p}B)/s
\end{tikzcd}
\end{subfigure}
\end{figure}

\noindent
The top-down arrow is our map \eqref{eq:fourth}. We move to the well-definition of the map \eqref{eq:fifth}.
\[
\frac{b + \textgoth{p}B}{s} = \frac{b' + \textgoth{p}B}{s'} 
\xRightarrow{?}
1/s \otimes (b + \textgoth{p}B) = 1/s' \otimes (b' + \textgoth{p}B)
\]
The left.
\[
u(s'(b + \textgoth{p}B) - s(b' + \textgoth{p}B)) = 0
\]
\[
us'b - usb' \in \textgoth{p}B
\]
The right.
\begin{align*}
\frac{1}{s'} \otimes (b' + \textgoth{p}B) 
& = \frac{us}{uss'} \otimes (b' + \textgoth{p}B) \\
& = \frac{1}{uss'} \otimes (usb' + \textgoth{p}B) \\
& = \frac{1}{uss'} \otimes (us'b + \textgoth{p}B) \\
& = \frac{us'}{uss'} \otimes (b + \textgoth{p}B) \\
& = \frac{1}{s} \otimes (b + \textgoth{p}B)
\end{align*}
The map \eqref{eq:fifth} is now well-defined. Additivity.
\begin{align*}
b + \textgoth{p}B + b' + \textgoth{p}B & = b + b' + \textgoth{p}B \\
& \mapsto 1/1 \otimes (b + b' + \textgoth{p}B) \\
& = 1/1 \otimes (b + \textgoth{p}B + b' + \textgoth{p}B) \\
& = 1/1 \otimes (b + \textgoth{p}B) + 1/1 \otimes (b' + \textgoth{p}B) \\
& = \ \mapsfrom (b + \textgoth{p}B) + \mapsfrom (b' + \textgoth{p}B)
\end{align*}
Multiplicativity.
\begin{align*}
(b + \textgoth{p}B)(b' + \textgoth{p}B) & = bb' + \textgoth{p}B \\
& \mapsto 1/1 \otimes (bb' + \textgoth{p}B) \\
& = (1/1)(1/1) \otimes (b + \textgoth{p}B)(b' + \textgoth{p}B) \\
& = (1/1 \otimes (b + \textgoth{p}B))(1/1 \otimes (b' + \textgoth{p}B)) \\
& = \ \mapsfrom (b + \textgoth{p}B) \mapsfrom (b' + \textgoth{p}B) 
\end{align*}

\[
\frac{ A_\textgoth{p} \otimes_A B }{ \textgoth{p}_\textgoth{p} \otimes_A B }
\]
\end{document}
