\documentclass{article}
\usepackage[utf8]{inputenc}
\usepackage{natbib}
\usepackage{graphicx}
\usepackage{tikz-cd}
\usepackage{amsmath}
\usepackage{amssymb}
\usepackage{amsthm}
\usepackage{yfonts}
\usepackage{hyperref}

\usepackage{geometry}
\geometry{
 a4paper,
 total={170mm,257mm},
 left=20mm,
 top=20mm,
 }

\hypersetup{
    colorlinks=true,
    linkcolor=blue,
    filecolor=magenta,      
    urlcolor=cyan,
}

\urlstyle{same}
\newtheorem*{fact}{Fact}

\title{Proposition 1.10}

 \begin{document}
\maketitle

\begin{center}
M.F. Atiyah, I.G. MacDonald \emph{Introduction to Commutative Algebra} \\
1 RINGS AND IDEALS
\end{center}

\vspace*{10px} 

\begin{fact}
If $z_1 \equiv 1 \pmod{\textgoth{a}}$ and $z_2 \equiv 1 \pmod{\textgoth{a}}$, then $z_1 z_2 \equiv 1 \pmod{\textgoth{a}}$ 
\end{fact}

$z_1 - 1 = a_1 \in \textgoth{a}$, $z_1 - 1 = a_1 \in \textgoth{a}$; $(z_1 - 1)(z_2 - 1) = a_1 a_3 \in \textgoth{a}$; $z_1 z_2 - z_1 - z_2 + 1 \in \textgoth{a}$; but $- z_2 + 1 \in \textgoth{a}$; now $z_1 z_2 - z_1 \in \textgoth{a}$; $z_1 z_2 \equiv z_1 \equiv 1 \pmod{\textgoth{a}}$; and the relation is transitive...

This fact is also because $A / \textgoth{a}$ is a ring: $z_1 z_2 + \textgoth{a} = (z_1 + \textgoth{a})(z_2 + \textgoth{a}) = (1 + \textgoth{a})(1 + \textgoth{a}) = 1 + \textgoth{a}$.

\vspace*{10px} 

Back in the proof of Proposition 1.10, $x_i = 1 - y_i \equiv 1 \pmod{\textgoth{a}_n}$; $\prod_{i = 1}^{n-1} x_i \equiv 1 \pmod{\textgoth{a}_n}$

\vspace*{10px} 

\begin{fact}
If for $y \in \textgoth{b}$, $y \equiv 1 \pmod{\textgoth{a}}$, then $\textgoth{a} + \textgoth{b} = (1)$.
\end{fact}

$y - 1 \in \textgoth{a}$; $y - 1 = x \ in \textgoth{a}$; $x + y = 1$. Now $\textgoth{a}, \textgoth{b}$ are coprime by the remark from the first paragraph on page 7.

\vspace*{10px} 

(ii) $\Rightarrow$: $\phi(x) = (1, 0, \dots, 0) = (1 + \textgoth{a}_1, 0 + \textgoth{a}_2, \dots, 0)$; $x + \textgoth{a}_1 = 1 + \textgoth{a}_1$ and $x + \textgoth{a}_1 = 0 + \textgoth{a}_1$; $x \equiv 1 \pmod{\textgoth{a}_1}$ and $x \equiv 0 \pmod{\textgoth{a}_2}$.

(ii) $\Leftarrow$: Same technique as in (i) to show that $x = \prod (1 - u_i) \equiv 1 \pmod{\textgoth{a}_1}$. As $x \in \textgoth{a}_1$ for $i > 1$, $x \equiv 0 \pmod{\textgoth{a}_1}$.

\end{document}S
