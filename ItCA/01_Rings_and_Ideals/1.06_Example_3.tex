\documentclass{article}
\usepackage[utf8]{inputenc}
\usepackage{natbib}
\usepackage{graphicx}
\usepackage{tikz-cd}
\usepackage{amsmath}
\usepackage{amssymb}
\usepackage{amsthm}
\usepackage{yfonts}
\usepackage{hyperref}

\usepackage{geometry}
\geometry{
 a4paper,
 total={170mm,257mm},
 left=20mm,
 top=20mm,
 }

\hypersetup{
    colorlinks=true,
    linkcolor=blue,
    filecolor=magenta,      
    urlcolor=cyan,
}

\urlstyle{same}
%\newcommand{\goth}[1]{\text{\textgoth{#1}}

\newtheorem*{fact}{Fact}

\title{Example 3}

\begin{document}

\maketitle

\begin{center}
M.F. Atiyah, I.G. MacDonald \emph{Introduction to Commutative Algebra}\\
1. RINGS AND IDEALS
\end{center}

\vspace*{10px} 

\begin{fact}
In a PID, a non-zero prime ideal is maximal.
\end{fact}

\vspace*{10px} 

Let $y \notin (x)$. Consider the ideal $(x, y) = (z)$. There is $x = az$, $y = bz$, and $cx + dy = z$; it follows that $acz + bdz = z$, and by cancellation, $ac + bd = 1$. Now $cx = z - dy = z - bdz = (1 - bd)z \in (x)$, so it must be one of $z \in (x)$ or $1 - bd \in (x)$; with the first, it would be $y \in (x)$, which is not true, so we stay with the second: $1 - bd \in (x)$. There is $x = a(cx + dy) = acx + ady$, then $ady \in (x)$, then $ad \in (x)$ or $y \in (x)$, but as the second is not true, $ad \in (x)$, whence $a \in (x)$ or $d \in (x)$. If it were $d \in (x)$, then we would have $bd \in (x)$, but $1 + (x) = bd + (x)$; so $d$ cannot be in $(x)$. We are left only with $a \in (x)$, now $a = a'x$, whence $x = az = aa'z$ and cancelling, $1 = aa'z$, then $z$ is a unit, and $(x, y) = (1)$. 


\end{document}