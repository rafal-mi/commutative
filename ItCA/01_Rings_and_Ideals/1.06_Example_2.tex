\documentclass{article}
\usepackage[utf8]{inputenc}
\usepackage{natbib}
\usepackage{graphicx}
\usepackage{tikz-cd}
\usepackage{amsmath}
\usepackage{amssymb}
\usepackage{yfonts}
\usepackage{hyperref}

\usepackage{geometry}
\geometry{
 a4paper,
 total={170mm,257mm},
 left=20mm,
 top=20mm,
 }

\hypersetup{
    colorlinks=true,
    linkcolor=blue,
    filecolor=magenta,      
    urlcolor=cyan,
}

\urlstyle{same}
%\newcommand{\goth}[1]{\text{\textgoth{#1}}

\title{Example 2}

 \begin{document}
\maketitle

\begin{center}
M.F. Atiyah, I.G. MacDonald \emph{Introduction to Commutative Algebra} \\
1 RINGS AND IDEALS
\end{center}

\vspace*{10px} 

$ A = k[x_1, \dots, x_n]$, $\textgoth{a} = (x_1, \dots, x_n)$. How is $\textgoth{a}^m$ the set of all polynomials with no terms of degree less than $m$?

In general, $\textgoth{a}^m$ is the set of sums of products $a_1 \cdots a_n$ where $a_i \in \textgoth{a}$. First, $\textgoth{a} = (x_1, \dots, x_m)$ is the set of finite sums
\[
f_1(x_1, \dots, x_n)x_1 + \cdots + f_n(x_1, \dots, x_n)x_n
\]

Then, $\textgoth{a}^m$ is is the set of finite sums of products
\begin{gather*}
(f_{11}(x_1, \dots, x_n)x_1 + \cdots + f_{1n}(x_1, \dots, x_n)x_n) \cdot \\
\cdot \; \cdots \; \cdot \\
\cdot (f_{m1}(x_1, \dots, x_n)x_1 + \cdots + f_{mn}(x_1, \dots, x_n)x_n)
\end{gather*}
Each such product has degree not less than $m$.

Now consider any polynomial of degree $m$. Each term is $x_1^{i_1} \cdots x_n^{i_n}$ with $i_1 + \cdots + i_n \geq m$. Now set $f_{11} = \dots = f_{i_1 1} = 1$ and $f_{1j} = \dots = f_{i_1 j} = 0$ for $j \neq 1$, then ...

\end{document}S
