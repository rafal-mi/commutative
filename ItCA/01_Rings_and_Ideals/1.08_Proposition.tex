\documentclass{article}
\usepackage[utf8]{inputenc}
\usepackage{natbib}
\usepackage{graphicx}
\usepackage{tikz-cd}
\usepackage{amsmath}
\usepackage{amssymb}
\usepackage{amsthm}
\usepackage{yfonts}
\usepackage{hyperref}

\usepackage{geometry}
\geometry{
 a4paper,
 total={170mm,257mm},
 left=20mm,
 top=20mm,
 }

\hypersetup{
    colorlinks=true,
    linkcolor=blue,
    filecolor=magenta,      
    urlcolor=cyan,
}

\urlstyle{same}
%\newcommand{\goth}[1]{\text{\textgoth{#1}}

\newtheorem*{fact}{Fact}

\title{Proposition 1.8}

\begin{document}

\maketitle

\begin{center}
M.F. Atiyah, I.G. MacDonald \emph{Introduction to Commutative Algebra}\\
1. RINGS AND IDEALS
\end{center}

\vspace*{10px} 

\begin{fact}
$a \in \textgoth{p} + (x) \wedge b \in \textgoth{p} + (y) \Rightarrow ab \in \textgoth{p} + (xy)$
\end{fact}

\vspace*{10px} 

$a = c + dx$ with $c \in \textgoth{p}$, $b = e + fy$ with $e \in \textgoth{p}$; then $ab = (c + dx)(e + fy) = ce + cfy + dex + dfy$; but first three are in $\textgoth{p}$ and last is in $(xy)$. 

\end{document}