\documentclass{article}
\usepackage[utf8]{inputenc}
\usepackage{natbib}
\usepackage{graphicx}
\usepackage{tikz-cd}
\usepackage{amsmath}
\usepackage{amssymb}
\usepackage{yfonts}
\usepackage{hyperref}

\usepackage{geometry}
\geometry{
 a4paper,
 total={170mm,257mm},
 left=20mm,
 top=20mm,
 }

\hypersetup{
    colorlinks=true,
    linkcolor=blue,
    filecolor=magenta,      
    urlcolor=cyan,
}

\urlstyle{same}
%\newcommand{\goth}[1]{\text{\textgoth{#1}}

\title{Exercise 1.06}

 \begin{document}
\maketitle

\begin{center}
M.F. Atiyah, I.G. MacDonald \emph{Introduction to Commutative Algebra}
\end{center}

\vspace*{10px} 

6. A ring $ A $ is such that every ideal not contained in the nilradical contains a nonzero idempotent (that is, an element $e$ such that $ e^2 = e \neq 0 $). Prove that the nilradical and the Jacobson radical of $A$ are equal.

\vspace*{10px} 

Assume that the ring has this property and has the Jacobson radical strictly larger than the nilradical. Now the J-radical must have nonzero idempotent: $ e^2 = e \neq 0 $. By Proposition 1.9 characterizing J-radicals, $ 1 - ey $ is a unit for any element $ y $ of the ring, especially $ 1 - e $ is a unit. Now we have $ e - e^2 = 0 $, then $ e(1-e) = 0 $. Multiplying by $ (1-e)^{-1} $ whe get $ e = 0 $, which is a contradiction.


\end{document}S
