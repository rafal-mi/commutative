\documentclass{article}
\usepackage[english]{babel}
\usepackage[center,uppercase,tiny]{titlesec}
\usepackage{amsfonts}
\usepackage{amsmath}
\usepackage{mathtools}
\newtheorem{theorem}{Theorem}[section]
\newtheorem{corollary}[theorem]{Corollary}
\newtheorem{lemma}[theorem]{Lemma}
\newtheorem{example}[theorem]{Example}

\title{
    \textit{Tensor Products II} \\
    {\small by Keith Conrad} \\
    Lecture Notes
}

\author{Rafal Mi}

\begin{document}

\maketitle

\section{Introduction}

\section{Tensor Products of Linear Maps}

\begin{example}
\end{example}

\begin{example}
\end{example}

\begin{theorem}
\end{theorem}

\begin{example}
\end{example}

\begin{theorem}
\end{theorem}

\section{Flat Modules}

\section{Tensor Products of Linear Maps and Base Extension}

\begin{theorem}
\end{theorem}

\begin{theorem}
\end{theorem}
In \textit{Modules over a PID}, Theorem 2.13, Corollary 2.15, and Theorem 4.2.

\begin{example}
\end{example}
In \textit{Tensor Products I}, Theorem 6.7, Example 6.8, Theorem 6.11.

\begin{theorem}
\end{theorem}
In \textit{Tensor Products I}, Theorem 6.7

\begin{example}
\end{example}

\begin{theorem}
\end{theorem}

\begin{example}
\end{example}

\begin{example}
\end{example}
A $ \mathbb{Z} / p^2 \mathbb{Z} $ -module homomorphism $ \phi: \mathbb{Z} / p\mathbb{Z} \rightarrow \mathbb{Z} / p^2 \mathbb{Z}$ is determined by its value on $1 = 1 + p \mathbb{Z}$ . Can it be a number $ 0 < k < p $ ? Then $ \phi(k + p - k) = \phi(k) + \phi(p - k) = k + p - k = p = p + p^2 \mathbb{Z} $, but in $\mathbb{Z}/p\mathbb{Z}$, the argument is $p = 0$ ($p + 3\mathbb{Z} = 0 + \mathbb{Z}$), then $\phi(0) = 0$ ($0 + p^2\mathbb{Z}$), a contradiction.

\begin{theorem}
\end{theorem}
For bottom vertical maps as isomorphism, see TP1, Corollary 6.27.
For field of fractions being a flat module see TP2, Theorem 3.3.
For tensor products of injective linear maps of vector spaces beinng injective, see TP2, Theorem 2.18; or TP2, Corollary 3.14. That $ M^{\otimes k} $ is free see TP1, Theorem 4.9. That over a domain. each free module is torsion-free, verify with a linear combination.

\section{Vector Spaces}

In paragraph 3, we can apply Theorem 3.2 twice: $ V \otimes W \xrightarrow{1 \otimes \psi} V \otimes W'\xrightarrow{\phi \otimes 1} V' \otimes W $.

\end{document}
